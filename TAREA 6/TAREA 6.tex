\documentclass[12pt,a4paper]{article}
\usepackage[latin1]{inputenc}
\usepackage[spanish]{babel}
\usepackage{amsmath}
\usepackage{amsfonts}
\usepackage{amssymb}
\usepackage{graphicx}
\usepackage[left=2cm,right=2cm,top=2cm,bottom=2cm]{geometry}
\author{luis martin santoyo mujica}
\title{TAREA 6}
\begin{document}
modulacion de ancho del pulso (PWM).
La modula cionpor ancho de pulsos tambi�n conocida como PWM por sus iniciales en ingl'ES(Pulse Width Modulation) De Una se~nalo fuente Delaware energ�a ' IAes una t ' ecnic una en la Que se modifica el ciclo de trabajo de una serie peri�dica mar senoidal o triangular por ejemplo. Este trabajo se Realiza ya el mar para transmitir informaci�n trav�s De la ware Naciones Unidas canal Delaware comunicaciones o paraca controlar la cantidad de energ�a What se env�a una carga. La construcci�n t�pica de Naciones Unidas circuito PWM se lleva una Cabo mediante un comparador con dos entradas y una salida. Una de las entradas se conecta a un oscilador de onda dientes de sierra, mientras que la otra queda disponible para la moduladora individual. En La Salida La Frecuencia es generalmente igual a la dela se�al dientes de sierra y el Ciclo de Trabajo est� en funci�n de la se�al portadora. La principal desventaja que presenta un los circuitos PWM es la posibilidad ad Delaware que haya interferencias generadas por radio frecuencia. �
Estas pueden minimizar ubicar el controlador cerca de la carga y realizar un filtro de la fuente de alimentaci�n. La modulaci�n panocha de pulsos es una t�cnica utilizada para regular la velocidad de giro de los motores el�ctricos de induce ieno asencronos, como lo requiere la presente par de fuerzas o giro en el motor constante y no supuno un desaprovechamiento a dela energ�a � � el�ctrica. Se                                                                                  utiliza tanto en corriente directa como en alternativa, como su nombre lo indica, al controlar: un momento alto (encendido o alimentado nta do) y Naciones Unidas Momento bajo(apa gado odesco nect ado)

 
Puente de diodos 
5.Regula dor de voltaje LM78 056.Regula dor de voltaje LM79 057.4 Capac ito res de 1
 �F 
Figura 1: Circuito de fuente sim� etricaPos terio rmen tese realiz �o la conexi�en el circuito generador de serietriangular visto en la ? gura 2,alimentando el ampli ? cadoroperacional de forma sim�etrica con+ - 5 V.Figura 2: Circuito generador de secci�ntriangularA continuaen la muestra del circuito con un amplificador operacional comparador el cual analiza unasecci�nde entrada respecto a un voltaje de referencia en la otra entrada [2]. Los Valores pico de la se~nal trian-gular se establecen en relaci'en una Los voltajes de umbral superiores.


\end{document}