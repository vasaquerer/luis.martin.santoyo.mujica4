\documentclass[12pt,a4paper]{article}
\usepackage[latin1]{inputenc}
\usepackage[spanish]{babel}
\usepackage{amsmath}
\usepackage{amsfonts}
\usepackage{amssymb}
\usepackage{graphicx}
\usepackage[left=2cm,right=2cm,top=2cm,bottom=2cm]{geometry}
\author{luis martin santoyo mujica}
\title{TAREA1}
\begin{document}

las caracter�sticas de convertidores:

los convertidores son elementos elementales capaces de alterar las caracter�sticas de la tensi�n y la corriente que recibe, transform�ndola de manera optimizada para los usos espec�ficos donde va a ser destinada en cada caso.

El avance de las telecomunicaciones en los �ltimos a�os tambi�n por lo cual a contribuido en aumento de equipos el�ctricos conectados en la red de distribuci�n el�ctrica de baja tensi�n.
Por lo cual los circuitos electr�nicos son alimentados con tensi�n continua.

los convertidores AC/AC:
-rectificador monof�sico de media onda.
-rectificador de onda completa.
-rectificador puente de onda completa.

El rectificador monof�sico de media onda:
es un circuito que convierte una se�al alterna en una se�al continua. La ventaja de este circuito es lo sencillo que es pero no se utiliza para unas industrias conduce durante el medio ciclo negativo el diodo esta en condici�n de bloqueo y el voltaje en la tabla es y ser� "0".

Transformador de onda completa:
el transformador de onda completa con transformador para operar como si fuera un rectificador de media onda.

El rectificador puente de onda completa:
este rectificador durante medio ciclo positivo del voltaje de enterad se suministra potencia a la larga a trav�s de D1 y D2 en el ciclo negativo los diodos D3 y D4 conducir�n, as� trabajando cada par de diodos rectificadores de media onda.

De esta forma se tiene un voltaje de corriente directa; sin embargo, esta es una variante que tiende a variar de un de repente.

Por otro lado, no todas las aplicaciones requieren un voltaje tan variante con un rizo Vsal.Por lo que es necesario proporcionar un voltaje con rizo peque�o, para ello es necesario utilizar un capacitor o condensador de salida el cual funciona como un filtro y tiende a mejorar el voltaje.


Convertidor CC:

Aun convertidor CC/CC se le puede considerar como un transformador de corriente AC ya que puede utilizarse como una fuente reductora o elevadora de voltaje.

CC/CC mas utilizados son los siguientes:
-Reductores.
-Elevadores.
-Cuk
-Reductores, Elevadores.

En el regulador reductor:
El voltaje de entrada es siempre mayor al voltaje de salida de ay al nombre de reductor este circuito funciona de dos modos 
   

\end{document}