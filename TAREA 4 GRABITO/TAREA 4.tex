\documentclass[12pt,a4paper]{article}
\usepackage[latin1]{inputenc}
\usepackage[spanish]{babel}
\usepackage{amsmath}
\usepackage{amsfonts}
\usepackage{amssymb}
\usepackage{graphicx}
\usepackage[left=2cm,right=2cm,top=2cm,bottom=2cm]{geometry}
\author{luis martin santoyo mujica}
\title{TAREA 4}
\begin{document}
Explicar arreglos y paera metros de los amplificadores clase
 
Amplificadores Clase B.

 Los amplificadores de clase B se caracterizan por tener intensidad casi nula a trav�s de sus transistores cuando no hay se�al en la entrada del circuito, por lo que en reposo el consumo es casi nulo. 
 
La caracter�stica principal de este tipo de amplificadores es el alto factor de amplificaci�n.
Amplificadores clase AB: Estos b�sicamente son la mezcla de los dos anteriores. Cuando el voltaje de polarizaci�n y la m�xima amplitud de la se�al entrante poseen valores que hacen que la corriente de salida circule durante menos del ciclo completo y m�s de la mitad del ciclo de la se�al de entrada, se les denomina: Amplificadores de potencia clase AB.
Dado que ocupa un lugar intermedio entre los de clase A y AB, cuando el voltaje de la se�al es moderado funciona como uno de clase A, cuando la se�al es fuerte se desempe�a como uno de clase B, con una eficiencia y deformaci�n moderadas.
 
Ventajas
?	Posee bajo consumo en reposo.
?	Aprovecha al m�ximo la Corriente entregada por la fuente.
?	Intensidad casi nula cuando est� en reposo.
Desventajas:

La caracter�stica principal de este tipo de amplificadores es el alto factor de amplificaci�n.
Amplificadores clase AB: Estos b�sicamente son la mezcla de los dos anteriores. Cuando el voltaje de polarizaci�n y la m�xima amplitud de la se�al entrante poseen valores que hacen que la corriente de salida circule durante menos del ciclo completo y m�s de la mitad del ciclo de la se�al de entrada, se les denomina: Amplificadores de potencia clase AB.
Dado que ocupa un lugar intermedio entre los de clase A y AB, cuando el voltaje de la se�al es moderado funciona como uno de clase A, cuando la se�al es fuerte se desempe�a como uno de clase B, con una eficiencia y deformaci�n moderadas.
            
Ventajas
?	Posee bajo consumo en reposo.
?	Aprovecha al m�ximo la Corriente entregada por la fuente.
?	Intensidad casi nula cuando est� en reposo.
Desventajas




\end{document}