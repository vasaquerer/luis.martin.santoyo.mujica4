\documentclass[12pt,a4paper]{article}
\usepackage[latin1]{inputenc}
\usepackage[spanish]{babel}
\usepackage{amsmath}
\usepackage{amsfonts}
\usepackage{amssymb}
\usepackage{graphicx}
\usepackage[left=2cm,right=2cm,top=2cm,bottom=2cm]{geometry}
\author{luis martin santoyo mujica}
\title{1 avance de proyecto}
\begin{document}
1- avance de proyecto.
Profesor: Carlos enrique moran garabito.
Universidad UPZMG polit�cnica zona metropolitana.         4-?A?
Jos� Guadalupe barrios S�nchez.
Luis Martin Santoyo Mujica.
Sistemas electr�nicos de interfaz.


 

Banda transportadora inteligente con rodillos.
Los transportadores de banda, son elementos important�simos dentro de las l�neas de producci�n ya que mueven de una forma continua paquetes, automatizan y hacen m�s f�cil y r�pido el trabajo. Funcionan mediante una banda rotatoria accionada de forma el�ctrica que puede transportar cajas, bandejas, productos o envases, entre otros.

Por lo cual la banda nos servir� para transportar cosas m�s sencilla y r�pida mente en la industria.



Prop�sito:
El alumno aprender� a ver las conexiones del PLC por medio de un sobware aprender� a desarrollar un proyecto con un PLC a nivel industrial.
	


Material:
PLC.
Rodillos.
madera.
Mezclilla.
Motor.
Sensores infrarojo.
Botonera.
Cables.
Relay?s.



Procedimiento:
ya sacando las medidas de la mezclilla ay que cortar conforme a la medida adecuada para realizarla y as� mismo poderla armar con los rodillos.
 



\end{document}